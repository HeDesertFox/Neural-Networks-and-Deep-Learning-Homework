\documentclass[notitlepage,cs4size,punct,oneside]{ctexrep}

% default paper settings, change it according to your word
\usepackage[a4paper,hmargin={2.54cm,2.54cm},vmargin={3.17cm,3.17cm}]{geometry}

\usepackage{amsmath,amssymb,amsthm}
\usepackage{titlesec}

% 公式编号的计数格式, 在章内计数
\numberwithin{equation}{chapter}

% set the abstract format, need abstract package

\usepackage[runin]{abstract}
\usepackage{algorithm}
\usepackage{algpseudocode}
\usepackage{booktabs}
\usepackage{graphicx}
\usepackage{float} % 图片和表格定位
\usepackage{subfigure} % 子图包
\usepackage{xcolor}





\usepackage[pdfborder={0 0 0},colorlinks=true,linkcolor=blue,CJKbookmarks=true, unicode=true]{hyperref}
%若要用LATEX编译, 请用下面的命令替代上述命令:
% \usepackage[dvipdfm,pdfborder={0 0 0},colorlinks=true,linkcolor=blue,CJKbookmarks=true]{hyperref}

\usepackage{mathrsfs}
\usepackage{longtable}
\usepackage{appendix} % 引入附录包
% 自定义附录标题格式
\newcommand{\appchapter}[2]{%
    \chapter*{附录~#1~~#2}%
    \addcontentsline{toc}{chapter}{附录~#1~~#2}%
    \phantomsection % 使得 hyperref 能够正确地引用到这个位置
    \label{app:#1} % 使用 label 来标记这个章节
}

\usepackage{listings} % 引入 listings 包以支持代码高亮
% 定义 Python 代码的样式
\lstset{
    language=Python,
    basicstyle=\ttfamily,
    keywordstyle=\bfseries\color{blue},
    commentstyle=\itshape\color{green!60!black},
    stringstyle=\color{orange},
    showstringspaces=false,
    breaklines=true,
    frame=single,
}

% 定义输出结果的样式
\lstdefinestyle{output}{
    basicstyle=\ttfamily,
    frame=single,
    breaklines=true,
}

%\usepackage{boondox-calo}
%\usepackage{dutchcal}

\setlength{\absleftindent}{1.5cm} \setlength{\absrightindent}{1.5cm}
\setlength{\abstitleskip}{-\parindent}
\setlength{\absparindent}{0cm}

% Theorem style
\newtheoremstyle{mystyle}{3pt}{3pt}{\kaishu}{0cm}{\bfseries}{}{1em}{}
\theoremstyle{mystyle}
\usepackage{listings}
\lstset{
    language = Python,
    backgroundcolor = \color{yellow!10},    % 背景色:淡黄
    basicstyle = \small\ttfamily,           % 基本样式 + 小号字体
    rulesepcolor= \color{gray},             % 代码块边框颜色
    breaklines = true,                  % 代码过长则换行
    numbers = left,                     % 行号在左侧显示
    numberstyle = \small,               % 行号字体
    showstringspaces   =   false,
    keywordstyle = \color{blue},            % 关键字颜色
    commentstyle =\color{gray},        % 注释颜色
    stringstyle = \color{red!100},          % 字符串颜色
    frame = shadowbox,                  % 用(带影子效果)方框框住代码块
    showspaces = false,                 % 不显示空格
    columns = fixed,                    % 字间距固定
    %escapeinside={<@}{@>}              % 特殊自定分隔符:<@可以自己加颜色@>
    morekeywords = {with},                % 自加新的关键字(必须前后都是空格)
    deletendkeywords = {compile}        % 删除内定关键字;删除错误标记的关键字用deletekeywords删!
}


\newtheorem{definition}{\hspace{2em}定义}[chapter]
% 如果没有章, 只有节, 把上面的[chapter]改成[section]
\newtheorem{theorem}[definition]{\hspace{2em}定理}
\newtheorem{axiom}[definition]{\hspace{2em}公理}
\newtheorem{lemma}[definition]{\hspace{2em}引理}
\newtheorem{proposition}[definition]{\hspace{2em}命题}
\newtheorem{corollary}[definition]{\hspace{2em}推论}
\newtheorem{remark}{\hspace{2em}注}[chapter]
%类似地定义其他“题头“. 这里“注“的编号与定义、定理等是分开的

\def\theequation{\arabic{chapter}.\arabic{equation}}
\def\thedefinition{\arabic{chapter}.\arabic{definition}.}

% title - \zihao{1} for size requirement \bfseries for font family requirement
\title{{\zihao{1}\bfseries 期末作业:自监督训练、Transformer与三维重建}}
\author{何益涵 \quad 20307110032\\吕文韬 \quad 23210180109}

\date{}
%%%%%%%%%%%%%%%%%%%导言区设置完毕
%%%%%%%%%%%%%%%%%%%%%%%%%%%%%%%%%%%%%%%%%%%%%%%%%%%%%%%%%%%%%%%%%%%%%
\begin{document}
%Styles for chapters/section
%若要将章标题左对齐, 用下面这个语句替换相应的设置
%\CTEXsetup[nameformat={\raggedright\zihao{3}\bfseries},%
\CTEXsetup[nameformat={\zihao{3}\bfseries},
           format={\zihao{3}\bfseries\centering},
           beforeskip={0.8cm}, afterskip={1.2cm}]{chapter}
\CTEXsetup[nameformat={\zihao{4}\bfseries},
           format={\zihao{4}\bfseries\centering},
           name={第~,~节},number={\arabic{section}},
           beforeskip={0.4cm},afterskip={0.4cm}]{section}
\CTEXsetup[nameformat={\zihao{-4}\bfseries},
           format={\zihao{-4}\bfseries},
           number={\arabic{section}.\arabic{subsection}.},
           beforeskip={0.4cm},afterskip={0.4cm}]{subsection}
\CTEXoptions[abstractname={摘要: }]
\CTEXoptions[bibname={\bfseries 参考文献}]

\renewcommand{\thepage}{\roman{page}}
\setcounter{page}{1}
% \tableofcontents\clearpage

\maketitle\renewcommand{\thepage}{\arabic{page}}
\thispagestyle{empty}\setcounter{page}{0}
%%%  论文的页码从正文开始计数, 摘要页不显示页码
% 撰写论文的摘要
\renewcommand{\abstractname}{摘要: }
\begin{abstract}
     本项目的仓库地址为\href{https://github.com/HeDesertFox/Neural-Networks-and-Deep-Learning-Homework-Group-Tasks.git}{该超链接}\footnote{\href{https://github.com/HeDesertFox/Neural-Networks-and-Deep-Learning-Homework-Group-Tasks.git}{https://github.com/HeDesertFox/Neural-Networks-and-Deep-Learning-Homework-Group-Tasks.git}}. 详见其中的文件夹\texttt{final}.其中包含任务1:自监督框架任务\texttt{task1\_self\_supervised\_learning}、任务2:Transformer任务\texttt{task2\_Transformer\_vs\_CNN}以及借助 NeRF 和 COLMAP完成的三维重建任务\texttt{task3\_object\_reconstruction\_view\_synthesis}.

     本项目提供了训练好的模型, 详见\href{https://pan.baidu.com/s/1P5qObkLj0eNHasfmEn_PVw?pwd=wwt1}{百度网盘}\footnote{链接:\url{https://pan.baidu.com/s/1P5qObkLj0eNHasfmEn_PVw?pwd=wwt1} 提取码:wwt1}, 若链接失效, 也可以通过 Github Issue 联系作者.

\end{abstract}



\chapter{对比监督学习和自监督学习在图像分类任务上的性能表现}
\section{任务描述}
\begin{enumerate}
\item 实现任一自监督学习算法并使用该算法在自选的数据集上训练ResNet-18, 随后在CIFAR-100数据集中使用Linear Classification Protocol对其性能进行评测;
\item 将上述结果与在ImageNet数据集上采用监督学习训练得到的表征在相同的协议下进行对比, 并比较二者相对于在CIFAR-100数据集上从零开始以监督学习方式进行训练所带来的提升;
\item 尝试不同的超参数组合, 探索自监督预训练数据集规模对性能的影响.
\end{enumerate}

\section{项目架构}
\begin{enumerate}
    \item \texttt{data\_preparation.py}文件:
    \item \texttt{model.py}文件:
    \item \texttt{training.py}文件:
    \item \texttt{test\_notebook.ipynb}文件:
    \end{enumerate}

\section{实验设置}
\subsection{模型选择}
本项目的自监督训练框架是 SimCLR, \href{http://proceedings.mlr.press/v119/chen20j/chen20j.pdf}{详见以下论文}.
\begin{figure}[H]
    \centering
    \includegraphics[scale=0.6]{simclr.png}
    \caption{SimCLR 架构}
    % \label{fig:fine}
\end{figure}
SimCLR 是一种用于自监督学习的框架, 旨在学习高质量的视觉表示. 其核心思想是通过对比学习来最大化正样本对之间的相似度, 并最小化负样本对之间的相似度. SimCLR架构主要包括以下几个关键步骤:
\begin{enumerate}
\item 数据增强:SimCLR对每张输入图像应用一系列随机的数据增强操作(如裁剪、旋转、颜色抖动等), 生成两种不同的视图. 这些视图构成正样本对, 而来自不同图像的视图构成负样本对.

\item 编码器:使用深度神经网络(通常是ResNet)作为编码器, 将每个增强后的图像视图映射到一个低维特征空间中. 编码器的输出是一个固定长度的特征向量.

\item 投影头(Projection Head):为了计算对比损失, SimCLR在编码器之后引入了一个小型的 MLP 投影头, 将编码器的输出特征向量进一步映射到另一个特征空间中. 这个步骤有助于学习更好的表示.

\item 对比损失(Contrastive Loss):SimCLR使用一种称为 NT-Xent(Normalized Temperature-scaled Cross Entropy)对比损失函数. 对于每对正样本对, 损失函数鼓励它们在特征空间中彼此接近, 同时使负样本对之间的距离尽可能远.
\end{enumerate}

整个 SimCLR 框架通过在一个大规模未标注数据集上进行训练, 逐步优化编码器和投影头的参数. 经过训练后, 编码器能够提取有用的图像特征, 这些特征可以用于下游任务(如分类、检测等).

SimCLR的一个显著优点是它仅依赖于数据增强和对比学习, 无需任何人工标注的数据, 从而有效利用大量未标注的数据进行训练. 通过学习更加通用的特征表示, SimCLR 在许多视觉任务中表现出色, 并成为自监督学习领域的重要方法之一.
\begin{figure}[H]
    \centering
    \includegraphics[scale=0.6]{simclr_code.png}
    \caption{SimCLR 伪代码}
    % \label{fig:fine}
\end{figure}

\subsection{数据集}
在本次期末作业的实验中, 我们选择了 CIFAR-10 数据集和 STL-10 数据集来预训练 ResNet18, 并在预训练好的 ResNet18 后加入二层 MLP 分类头, 并在 CIFAR-100 数据集上进行了微调.

\subsection{优化器和超参数}
本次实验使用 AdamW 做为优化器, 学习率为 $0.0003$, 设置批次大小为 512. 在 CIFAR10 数据集上, 我们预训练了 90 个 epochs, 在 STL10 数据集上, 我们预训练了 70 个 epochs, 得到了两个版本的 ResNet18.

在训练中, 我们的温度参数设置为 0.07, 特征投射头的维度为 128.

\subsection{数据增强}
在本次实验中, 我们选择了和论文中相同的数据增强方法来生成正样本对, 我们对图像按照以下顺序进行数据增强:对图像进行随机裁剪:对图像进行随机裁剪、对图像进行随机色彩失真, 对图像进行高斯模糊.


\section{实验结果}
\subsection{数据分析}

我们比较了几组 ResNet18 在下游 CIFAR100 分类任务上的性能. 以下是准确率曲线, 四条曲线由上到下分别是在 \textcolor{orange}{ImageNet 上预训练}的 ResNet18, 在 \textcolor{red}{CIFAR10 上使用 SimCLR 预训练}的 ResNet18, 在 \textcolor{green}{STL10 上 使用 SimCLR 预训练}的 ResNet18 和\textcolor{purple}{随机初始化}的 ResNet18. 可以看到, 预训练的模型在分类任务上微调时很快收敛, 详见下图:
\begin{figure}[H]
    \centering
    \includegraphics[width=1\linewidth]{finetune_accuracy.png}
    \caption{不同预训练策略下的ResNet18在分类任务上的性能比较}
    \label{finetune_accuracy}
\end{figure}

下图是四个模型的微调损失函数曲线, 由下到上分别是在 \textcolor{orange}{ImageNet 上预训练}的 ResNet18, 在 \textcolor{red}{CIFAR10 上使用 SimCLR 预训练}的 ResNet18, 在 \textcolor{green}{STL10 上 使用 SimCLR 预训练}的 ResNet18 和\textcolor{purple}{随机初始化}的 ResNet18.:

\begin{figure}[H]
    \centering
    \includegraphics[width=1\linewidth]{fine-tune_loss.png}
    \caption{微调任务的损失函数曲线}
    \label{fig:enter-label}
\end{figure}

\subsection{结论}

通过以上实验, 我们发现使用自监督训练方法预训练得到的 ResNet18 相比于随机初始化的 ResNet18 在具体任务微调上具有更为显著的优势.

但同时, 并非是预训练数据集越大越好, STL10 的数据集体量远大于 CIFAR10, 但两个预训练得到的模型在 CIFAR100 分类任务上表现接近, 甚至 CIFAR10 预训练得到的模型在某些 epochs 的评测中效果要好于 STL10, 推测有以下两个可能的原因:

\begin{enumerate}
    \item SimCLR 的特征投射头设置较小, CIFAR10 的数据量已经达到模型承载极限.
    \item SimCLR 损失函数存在优化空间.
\end{enumerate}

\chapter{在 CIFAR-100 数据集上比较基于 Transformer 和 CNN 的图像分类模型}
\section{任务描述}
\begin{enumerate}
\item 分别基于 CNN 和 Transformer 架构实现具有相近参数量的图像分类网络;
\item 在 CIFAR-100 数据集上采用相同的训练策略对二者进行训练, 其中数据增强策略中应包含 CutMix;
\item 尝试不同的超参数组合, 尽可能提升各架构在CIFAR-100上的性能以进行合理的比较.
\end{enumerate}

\section{项目架构}
此任务的所有文件在\texttt{task2\_Transformer\_vs\_CNN}文件夹中, 其中包含以下文件:
\begin{enumerate}
\item \texttt{data\_preparation.py}文件:包含数据下载函数与预处理函数, 其中包含 Cutmix 的实现.
\item \texttt{model.py}文件:包含 Resnet 和 ViT 的模型构造函数.
\item \texttt{training.py}文件:包含训练和超参数调优函数.
\item \texttt{test\_notebook.ipynb}文件:包含 Transformer 架构和 CNN 架构的对比任务的全部流程, 如数据加载、调参和训练可视化.
\item \texttt{para\_count.py}文件:精细统计模型每一层的参数量.
\end{enumerate}

\section{实验设置}
\subsection{数据增广}
此项目的数据增广在常规图像变化的基础上增加了 Cutmix 增强方法. \href{http://openaccess.thecvf.com/content_ICCV_2019/papers/Yun_CutMix_Regularization_Strategy_to_Train_Strong_Classifiers_With_Localizable_Features_ICCV_2019_paper.pdf}{详见以下论文}.

CutMix是一种数据增强技术, 旨在提升网络的泛化性能. CutMix的主要思想是在训练图像之间剪切并粘贴图像块, 同时按比例混合相应的标签. 此方法有助于模型从更多样化的训练样本中学习, 从而提高其鲁棒性和性能.

CutMix的原理包括两个主要步骤:从一张图像中剪切出一个块并将其粘贴到另一张图像上, 然后按比例调整两张图像的标签. 具体来说, 对于两张图像 $x_A$ 和 $x_B$ 及其对应的标签 $y_A$ 和 $y_B$, 新的图像 $\tilde{x}$ 和标签 $\tilde{y}$ 的生成方式如下:
\begin{equation}
\tilde{x} = M \odot x_A + (1 - M) \odot x_B
\end{equation}
\begin{equation}
\tilde{y} = \lambda y_A + (1 - \lambda) y_B
\end{equation}
其中, $M$ 是一个二进制掩码, 表示图像块的应用位置, $\odot$ 表示元素级乘法, $\lambda \in [0, 1]$ 是表示原始图像保留比例的随机变量, 通常从Beta分布中采样.
\begin{figure}[H]
    \centering
    \includegraphics[scale=0.7]{cutmix.png}
    \caption{Cutmix与其他混合增强方法的对比}
    % \label{fig:acc}
\end{figure}

CutMix已被证明能通过以下几方面增强深度学习模型的性能:
\begin{itemize}
    \item 提高泛化性能:通过使模型暴露于更广泛的图像变体, CutMix有助于防止过拟合, 从而提高在未见数据上的泛化性能.
    \item 正则化:该技术类似于dropout, 通过引入随机性防止模型过度依赖特定特征, 起到正则化的作用.
    \item 标签平滑:标签混合有助于模型学习更柔和的标签, 这对于分类任务中类间界限不明确的情况尤为有益.
\end{itemize}
实验证明, 与传统的数据增强技术相比, 使用CutMix训练的模型在准确率和对抗攻击的鲁棒性方面表现更佳.

CutMix的实现包括以下步骤:
\begin{enumerate}
    \item 随机选择两张图像:从训练集中选择两张图像 $x_A$ 和 $x_B$ 及其对应的标签 $y_A$ 和 $y_B$.
    \item 生成二进制掩码:通过在第一张图像中选择一个随机矩形区域, 并将该区域对应的第二张图像的部分设为零, 创建一个二进制掩码 $M$.
    \item 应用CutMix:使用二进制掩码组合两张图像, 创建新的训练样本. 根据掩码区域的面积按比例调整标签.
    \item 更新训练数据:使用新的图像和标签作为训练过程的一部分.
\end{enumerate}

在本项目的实现中, 函数 \texttt{rand\_bbox} 用于生成随机边界框, 而 \texttt{cutmix} 函数则将 CutMix 增强应用于一批图像. 在训练函数每次载入小批量数据时, 我们调用\texttt{cutmix} 函数.


\subsection{模型选择}
本项目采用的CNN架构是 resnet 架构, 具体采用了 resnet18 架构, 也可以改为更大的 resnet 架构.

本项目采用的 Transformer 架构是经典的 ViT(Vision Transformer) 架构. \href{https://arxiv.org/pdf/2010.11929.pdf?fbclid=IwAR1NafJDhZjkARvCswpV6kS9_hMa0ycvzwhlCb7cqAGwgzComFXcScxgA8o}{详见以下论文}:
\begin{figure}[H]
    \centering
    \includegraphics[scale=0.55]{ViT.png}
    \caption{ViT 架构}
    % \label{fig:fine}
\end{figure}
ViT 是一种将 Transformer 应用于图像分类任务的新型架构. Transformer 最初在自然语言处理(NLP)任务中取得了巨大成功, ViT则将其应用于计算机视觉领域, 通过将图像划分为一系列图像块(patch), 并将这些图像块视为序列数据进行处理. ViT 在许多图像分类任务上表现出色, 特别是在大型数据集上.

ViT 的核心思想是将图像分割成固定大小的图像块, 然后将这些图像块展平并嵌入到更高维度的向量空间中. 每个图像块通过线性变换被映射为一个特征向量, 这些特征向量连同位置编码一起作为 Transformer 编码器的输入.

具体步骤如下:
\begin{enumerate}
    \item 图像分割:将输入图像 $x \in \mathbb{R}^{H \times W \times C}$ 划分为 $N$ 个不重叠的图像块, 每个图像块的大小为 $P \times P$, 因此 $N = \frac{HW}{P^2}$.
    \item 图像块展平和嵌入:将每个图像块展平并通过线性变换映射到$d$维特征向量, 得到 $z_0 \in \mathbb{R}^{N \times d}$.
    \item 位置编码:为每个图像块添加位置编码, 以保留空间位置信息, 位置编码与特征向量相加形成最终输入 $z_0 + E_{pos}$.
    \item Transformer 编码器:将上述输入序列传递给标准的 Transformer 编码器, 编码器由多层自注意力机制和前馈神经网络组成.
    \item 分类:使用分类标记(class token)作为输入序列的一部分, 通过 Transformer 编码器的输出进行图像分类.
\end{enumerate}
ViT 在大型数据集上的表现优异, 并且在参数数量相近的情况下, 能够比传统卷积神经网络取得更好的性能. 其主要优势包括:
\begin{itemize}
    \item 灵活性:ViT 能够处理不同分辨率和大小的图像, 只需调整图像块的大小即可.
    \item 全局信息:自注意力机制能够捕捉图像中的全局信息, 而不是局限于局部感受野.
    \item 可扩展性:ViT 能够轻松扩展到更大的模型尺寸和更多的参数, 从而进一步提高性能.
\end{itemize}
然而, ViT的一个主要挑战是在小型数据集上容易过拟合, 因此在使用 ViT 时, 通常需要预训练模型或更强的数据增强技术(如 Cutmix).

\subsection{模型参数量估计}
ResNet的分类头输出维数是100, resnet18模型的参数量统计为:11227812.

在我们的 ViT 模型中, \texttt{image\_size=32} 表示输入图像的尺寸为 32x32 像素;\texttt{patch\_size=4} 表示将图像划分为 4x4 像素的 patch, 因此一个 32x32 的图像将被划分为 64 个 patch;\texttt{num\_classes=100} 表示模型需要进行100类分类;\texttt{dim=192}表示每个patch嵌入到一个192维的向量空间;\texttt{depth=25}表示模型包含25层Transformer编码器块;\texttt{heads=6}表示每个自注意力层包含6个独立的注意力头;\texttt{mlp\_dim=384}表示前馈神经网络的隐藏层维度为384. ViT 模型的参数统计为:11139844.

更细致的统计见附录A.

\subsection{优化器设置}
两种架构的优化器都选择 Adam 优化器. 除学习率和权重衰退以外, 其余参数都为默认参数.


\section{调参结果}
本实验调节两个参数以尽可能提高模型性能:学习率 \texttt{lr} 和权重衰退 \texttt{weight\_decay}.

经过多轮调参, 最终确定 resnet18 的 \texttt{lr} 为1e-3, \texttt{weight\_decay} 为0. ViT 的 \texttt{lr} 为4e-4, \texttt{weight\_decay} 为1e-5.


\section{实验结果}
\subsection{数据分析}
以下图片中, \textcolor{orange}{橙线}都是 \textcolor{orange}{resnet18 网络}的数据曲线, \textcolor{blue}{蓝线}都是 \textcolor{blue}{ViT 网络}的数据曲线.

\begin{figure}[H]
    \centering
    \includegraphics[scale=0.75]{transformer_acc.png}
    \caption{准确率对比}
    % \label{fig:acc}
\end{figure}
可以看出两个模型的准确率在训练的过程中不断上升, 并且都没有饱和, 理论上如果训练更长时间, 两者都能获得更好的效果. 训练初始阶段 resnet18 性能更好, 约30个 epoch 左右被 ViT 超越. resnet18 的最终验证集准确率为57.88\%, ViT的最终验证集准确率为65.53\%.
\begin{figure}[H]
    \centering
    \includegraphics[scale=0.75]{transformer_loss.png}
    \caption{损失函数对比}
    % \label{fig:loss}
\end{figure}
两个模型的损失函数值也逐渐下降, 初始时 resnet18 下降更快, 但随后被 ViT 超越.



\subsection{结论}
以上实验结果一定程度上验证了 ViT 原论文中的结论, 即 ViT 虽然在较小的数据集上性能不如 CNN 网络, 但在更大的数据集上, ViT 的性能可以超越 CNN 架构. 在我们的实验中, 可以认为训练轮数较少等价于数据集较小, 因此随着训练的进行, 模型学习的数据量增多, ViT 的性能逐渐超越 resnet18.

以下是 ViT 原论文中两种模型的性能对比, 其中 BiT 是一种基于CNN的模型架构, 灰色区域是其架构的性能范围, 随着数据量的增加, ViT 的性能增加比BiT更多.

\begin{figure}[H]
    \centering
    \includegraphics[scale=0.75]{ViT_vs_CNN.png}
    \caption{数据集大小与模型性能的关系}
    % \label{fig:loss}
\end{figure}

ViT 由于其较复杂的结构, 在小数据集和短期训练中容易过拟合, 导致其性能不如 resnet18. 但随着训练数据量的增加, ViT 能够更好地利用其全局注意力机制, 捕捉数据中的长距离依赖关系, 从而逐渐提升其性能. 在相同训练策略下, ViT 需要更多的训练时间和数据量来展现其优势. 因此, 在实际应用中, 为了充分发挥 ViT 的潜力, 通常需要结合大规模数据集或进行预训练. 对于数据量有限的任务, resnet18 等传统 CNN 架构依然是有效的选择. 而对于数据量充足的任务, ViT 则展现出更强的性能和泛化能力.

\chapter{基于 NeRF 的物体重建和新视图合成}
\section{任务描述}
\begin{enumerate}
\item 选取身边的物体拍摄多角度图片/视频, 并使用COLMAP估计相机参数, 随后使用现成的框架进行训练;
\item 基于训练好的NeRF渲染环绕物体的视频, 并在预留的测试图片上评价定量结果.
\end{enumerate}

\section{项目架构}
本次实验使用 nerfstudio 完成了工程部分, 选择的模型是 nerfacto, 是 nerfstudio 的默认选项, 关于模型和 nerfstudio 的架构细节, 详见\href{https://arxiv.org/pdf/2302.04264}{此论文}.

本次实验使用了一段 67s 的视频做为训练数据集, COLMAP 将视频划分为了 335 帧, 其中 90\% 的数据被用作模型的训练, 10\% 的数据被用作模型的测试.

\section{NeRF 介绍}
NeRF(Neural Radiance Fields)是一种使用神经网络来表示三维场景的技术. 其核心思想是通过输入空间位置和视角方向, 利用神经网络预测该位置的颜色和体积密度. NeRF的主要原理包括以下几个方面:
\subsection{体积渲染}
体积渲染是NeRF的基础, 通过对三维空间中的点进行采样来生成图像. 具体步骤如下:
\begin{enumerate}
    \item 对每条摄像机射线(ray)进行均匀采样, 得到多个三维采样点.
    \item 对每个采样点, 输入其空间坐标和视角方向到神经网络中, 预测该点的颜色和密度.
    \item 通过体积渲染方程, 结合所有采样点的颜色和密度, 计算出该射线对应的像素颜色.
\end{enumerate}
\subsection{网络架构}
NeRF的神经网络通常是一个多层感知器(MLP), 其输入是空间坐标和视角方向, 输出是颜色和体积密度. 网络架构的细节如下:
\begin{itemize}
    \item 输入: 三维坐标 $\mathbf{x} = (x, y, z)$ 和视角方向 $\mathbf{d} = (d_x, d_y, d_z)$.
    \item 输出: 体积密度 $\sigma$ 和颜色 $\mathbf{c} = (r, g, b)$.
    \item 网络层数: 通常包含8层, 每层包含256个神经元, 使用ReLU激活函数.
    \item 跳跃连接: 在中间层添加跳跃连接, 以保留高频信息.
\end{itemize}
\subsection{优化}
NeRF通过优化网络参数, 使得合成图像与真实图像之间的误差最小.

使用NeRF进行三维重建的流程包括数据准备、网络初始化、训练和渲染. 具体步骤如下:
\subsection{数据准备}
为了训练NeRF模型, 需要多视角的图像数据和相机参数. 数据准备过程如下:
\begin{enumerate}
    \item 收集多视角图像:从不同视角拍摄目标场景的图像, 确保覆盖足够的视角范围.
    \item 获取相机参数:记录每张图像对应的相机位置和方向, 通常使用相机标定技术获取这些参数.
\end{enumerate}

\subsection{网络初始化}
初始化NeRF模型, 包括网络参数和优化器. 具体步骤如下:
\begin{enumerate}
    \item 定义网络结构:根据NeRF的架构定义MLP模型.
    \item 初始化参数:随机初始化网络参数.
    \item 配置优化器:通常使用Adam优化器, 并设置学习率等超参数.
\end{enumerate}

\subsection{训练}
训练NeRF模型, 使其能够准确预测三维场景中的颜色和密度. 训练过程如下:
\begin{enumerate}
    \item 射线采样:对于每张图像中的每个像素, 从相机中心沿视线方向采样多条射线.
    \item 空间采样:对于每条射线, 在其沿途均匀采样多个三维点.
    \item 颜色和密度预测:将每个采样点的坐标和视角方向输入神经网络, 预测该点的颜色和密度.
    \item 体积渲染:根据所有采样点的颜色和密度, 通过体积渲染方程计算每条射线的像素颜色.
    \item 损失计算:计算合成图像与真实图像之间的L2损失.
    \item 参数更新:通过反向传播和梯度下降算法, 更新网络参数.
\end{enumerate}
\subsection{渲染}
训练完成后, 使用优化后的NeRF模型生成新视角的图像. 渲染过程如下:
\begin{enumerate}
    \item 定义视角:选择新视角的相机位置和方向.
    \item 射线采样:从新视角的相机位置沿视线方向采样射线.
    \item 空间采样:在每条射线上均匀采样多个三维点.
    \item 颜色和密度预测:将每个采样点的坐标和视角方向输入神经网络, 预测该点的颜色和密度.
    \item 体积渲染:根据所有采样点的颜色和密度, 通过体积渲染方程计算每条射线的像素颜色, 生成新视角的图像.
\end{enumerate}

NeRF通过神经网络有效地表示三维场景中的颜色和密度, 结合体积渲染技术, 实现了高质量的三维重建. 这种方法不仅在保留细节方面表现出色, 而且在计算机视觉和图形学领域得到了广泛应用.



\section{实验设置}
本次实验借助了架构 Nerf Studio, 此框架提供了一键式训练一些现有 nerf 模型的环境. 实验的主要设置为:
\begin{enumerate}
    \item 优化器:采用 Adam 优化器, NeRF 的训练有三部分:相机参数优化、场和建议网络, 这几部分的学习率中, 后两个的学习率我们使得其线性下降, 第一个学习率我们令其以 $t^{-1}$ 的速度下降, 最低为 $1\times 10^{-4}$.

    \item 我们训练了 100 个 epoch, 每个 epoch 都使用了 4096 的 batch size, 每 500 次迭代之后评估一张测试集的图片, 25000 个迭代之后评估所有图片.
\end{enumerate}


\section{实验结果}
本次实验的训练损失函数和测试集上的损失函数见下图:
\begin{figure}[H]
    \centering
    \includegraphics[width=1\linewidth]{nerfloss.png}
    \caption{NeRFacto 训练损失函数}
    \label{fig:enter-label}
\end{figure}

\begin{figure}[H]
    \centering
    \includegraphics[width=1\linewidth]{eval_loss.png}
    \caption{NeRFacto 测试损失函数}
    \label{fig:enter-label}
\end{figure}

同时, 我们还输出了在测试图像上的 PSNR, 见下图:
\begin{figure}[H]
    \centering
    \includegraphics[width=1\linewidth]{psnr.png}
    \caption{测试图片上的 PSNR}
    \label{fig:enter-label}
\end{figure}

\subsection{结果的可视化}

我们使用 NeRF 训练的结果重建了 3D 点云, 并渲染了环绕物体的视频.

我们选取的物体是一个可回收垃圾桶, 其放置的环境在光华楼东主楼 14 楼的电梯间, 3D点云重建的效果如下:

\begin{figure}[H]
    \centering
    \includegraphics[width=1\linewidth]{3drecon.png}
    \caption{3D 重建的可视化}
    \label{fig:enter-label}
\end{figure}

我们渲染得到的视频在电子文件中可用.
\newpage
% 附录开始
\begin{appendices}
    \appchapter{A}{ResNet18 和 ViT 的模型参数量精细统计}
可以运行\texttt{task2\_Transformer\_vs\_CNN}文件夹中的\texttt{para\_count.py}文件来统计模型参数. 以下是统计结果:
\small{
\begin{lstlisting}[style=output]
ResNet-18 Architecture and Parameters:
Layer                                    Shape              Params
==================================================================
conv1.weight                             [64, 3, 7, 7]      9408
bn1.weight                               [64]               64
bn1.bias                                 [64]               64
layer1.0.conv1.weight                    [64, 64, 3, 3]     36864
layer1.0.bn1.weight                      [64]               64
layer1.0.bn1.bias                        [64]               64
layer1.0.conv2.weight                    [64, 64, 3, 3]     36864
layer1.0.bn2.weight                      [64]               64
layer1.0.bn2.bias                        [64]               64
layer1.1.conv1.weight                    [64, 64, 3, 3]     36864
layer1.1.bn1.weight                      [64]               64
layer1.1.bn1.bias                        [64]               64
layer1.1.conv2.weight                    [64, 64, 3, 3]     36864
layer1.1.bn2.weight                      [64]               64
layer1.1.bn2.bias                        [64]               64
layer2.0.conv1.weight                    [128, 64, 3, 3]    73728
layer2.0.bn1.weight                      [128]              128
layer2.0.bn1.bias                        [128]              128
layer2.0.conv2.weight                    [128, 128, 3, 3]   147456
layer2.0.bn2.weight                      [128]              128
layer2.0.bn2.bias                        [128]              128
layer2.0.downsample.0.weight             [128, 64, 1, 1]    8192
layer2.0.downsample.1.weight             [128]              128
layer2.0.downsample.1.bias               [128]              128
layer2.1.conv1.weight                    [128, 128, 3, 3]   147456
layer2.1.bn1.weight                      [128]              128
layer2.1.bn1.bias                        [128]              128
layer2.1.conv2.weight                    [128, 128, 3, 3]   147456
layer2.1.bn2.weight                      [128]              128
layer2.1.bn2.bias                        [128]              128
layer3.0.conv1.weight                    [256, 128, 3, 3]   294912
layer3.0.bn1.weight                      [256]              256
layer3.0.bn1.bias                        [256]              256
layer3.0.conv2.weight                    [256, 256, 3, 3]   589824
layer3.0.bn2.weight                      [256]              256
layer3.0.bn2.bias                        [256]              256
layer3.0.downsample.0.weight             [256, 128, 1, 1]   32768
layer3.0.downsample.1.weight             [256]              256
layer3.0.downsample.1.bias               [256]              256
layer3.1.conv1.weight                    [256, 256, 3, 3]   589824
layer3.1.bn1.weight                      [256]              256
layer3.1.bn1.bias                        [256]              256
layer3.1.conv2.weight                    [256, 256, 3, 3]   589824
layer3.1.bn2.weight                      [256]              256
layer3.1.bn2.bias                        [256]              256
layer4.0.conv1.weight                    [512, 256, 3, 3]   1179648
layer4.0.bn1.weight                      [512]              512
layer4.0.bn1.bias                        [512]              512
layer4.0.conv2.weight                    [512, 512, 3, 3]   2359296
layer4.0.bn2.weight                      [512]              512
layer4.0.bn2.bias                        [512]              512
layer4.0.downsample.0.weight             [512, 256, 1, 1]   131072
layer4.0.downsample.1.weight             [512]              512
layer4.0.downsample.1.bias               [512]              512
layer4.1.conv1.weight                    [512, 512, 3, 3]   2359296
layer4.1.bn1.weight                      [512]              512
layer4.1.bn1.bias                        [512]              512
layer4.1.conv2.weight                    [512, 512, 3, 3]   2359296
layer4.1.bn2.weight                      [512]              512
layer4.1.bn2.bias                        [512]              512
fc.weight                                [100, 512]         51200
fc.bias                                  [100]              100
==================================================================
Total Parameters: 11227812

ViT Architecture and Parameters:
Layer                                    Shape              Params
==================================================================
pos_embedding                            [1, 65, 192]       12480
cls_token                                [1, 1, 192]        192
to_patch_embedding.1.weight              [48]               48
to_patch_embedding.1.bias                [48]               48
to_patch_embedding.2.weight              [192, 48]          9216
to_patch_embedding.2.bias                [192]              192
to_patch_embedding.3.weight              [192]              192
to_patch_embedding.3.bias                [192]              192
transformer.norm.weight                  [192]              192
transformer.norm.bias                    [192]              192
transformer.layers.0.0.norm.weight       [192]              192
transformer.layers.0.0.norm.bias         [192]              192
transformer.layers.0.0.to_qkv.weight     [1152, 192]        221184
transformer.layers.0.0.to_out.0.weight   [192, 384]         73728
transformer.layers.0.0.to_out.0.bias     [192]              192
transformer.layers.0.1.net.0.weight      [192]              192
transformer.layers.0.1.net.0.bias        [192]              192
transformer.layers.0.1.net.1.weight      [384, 192]         73728
transformer.layers.0.1.net.1.bias        [384]              384
transformer.layers.0.1.net.4.weight      [192, 384]         73728
transformer.layers.0.1.net.4.bias        [192]              192
transformer.layers.1.0.norm.weight       [192]              192
transformer.layers.1.0.norm.bias         [192]              192
transformer.layers.1.0.to_qkv.weight     [1152, 192]        221184
transformer.layers.1.0.to_out.0.weight   [192, 384]         73728
transformer.layers.1.0.to_out.0.bias     [192]              192
transformer.layers.1.1.net.0.weight      [192]              192
transformer.layers.1.1.net.0.bias        [192]              192
transformer.layers.1.1.net.1.weight      [384, 192]         73728
transformer.layers.1.1.net.1.bias        [384]              384
transformer.layers.1.1.net.4.weight      [192, 384]         73728
transformer.layers.1.1.net.4.bias        [192]              192
transformer.layers.2.0.norm.weight       [192]              192
transformer.layers.2.0.norm.bias         [192]              192
transformer.layers.2.0.to_qkv.weight     [1152, 192]        221184
transformer.layers.2.0.to_out.0.weight   [192, 384]         73728
transformer.layers.2.0.to_out.0.bias     [192]              192
transformer.layers.2.1.net.0.weight      [192]              192
transformer.layers.2.1.net.0.bias        [192]              192
transformer.layers.2.1.net.1.weight      [384, 192]         73728
transformer.layers.2.1.net.1.bias        [384]              384
transformer.layers.2.1.net.4.weight      [192, 384]         73728
transformer.layers.2.1.net.4.bias        [192]              192
transformer.layers.3.0.norm.weight       [192]              192
transformer.layers.3.0.norm.bias         [192]              192
transformer.layers.3.0.to_qkv.weight     [1152, 192]        221184
transformer.layers.3.0.to_out.0.weight   [192, 384]         73728
transformer.layers.3.0.to_out.0.bias     [192]              192
transformer.layers.3.1.net.0.weight      [192]              192
transformer.layers.3.1.net.0.bias        [192]              192
transformer.layers.3.1.net.1.weight      [384, 192]         73728
transformer.layers.3.1.net.1.bias        [384]              384
transformer.layers.3.1.net.4.weight      [192, 384]         73728
transformer.layers.3.1.net.4.bias        [192]              192
transformer.layers.4.0.norm.weight       [192]              192
transformer.layers.4.0.norm.bias         [192]              192
transformer.layers.4.0.to_qkv.weight     [1152, 192]        221184
transformer.layers.4.0.to_out.0.weight   [192, 384]         73728
transformer.layers.4.0.to_out.0.bias     [192]              192
transformer.layers.4.1.net.0.weight      [192]              192
transformer.layers.4.1.net.0.bias        [192]              192
transformer.layers.4.1.net.1.weight      [384, 192]         73728
transformer.layers.4.1.net.1.bias        [384]              384
transformer.layers.4.1.net.4.weight      [192, 384]         73728
transformer.layers.4.1.net.4.bias        [192]              192
transformer.layers.5.0.norm.weight       [192]              192
transformer.layers.5.0.norm.bias         [192]              192
transformer.layers.5.0.to_qkv.weight     [1152, 192]        221184
transformer.layers.5.0.to_out.0.weight   [192, 384]         73728
transformer.layers.5.0.to_out.0.bias     [192]              192
transformer.layers.5.1.net.0.weight      [192]              192
transformer.layers.5.1.net.0.bias        [192]              192
transformer.layers.5.1.net.1.weight      [384, 192]         73728
transformer.layers.5.1.net.1.bias        [384]              384
transformer.layers.5.1.net.4.weight      [192, 384]         73728
transformer.layers.5.1.net.4.bias        [192]              192
transformer.layers.6.0.norm.weight       [192]              192
transformer.layers.6.0.norm.bias         [192]              192
transformer.layers.6.0.to_qkv.weight     [1152, 192]        221184
transformer.layers.6.0.to_out.0.weight   [192, 384]         73728
transformer.layers.6.0.to_out.0.bias     [192]              192
transformer.layers.6.1.net.0.weight      [192]              192
transformer.layers.6.1.net.0.bias        [192]              192
transformer.layers.6.1.net.1.weight      [384, 192]         73728
transformer.layers.6.1.net.1.bias        [384]              384
transformer.layers.6.1.net.4.weight      [192, 384]         73728
transformer.layers.6.1.net.4.bias        [192]              192
transformer.layers.7.0.norm.weight       [192]              192
transformer.layers.7.0.norm.bias         [192]              192
transformer.layers.7.0.to_qkv.weight     [1152, 192]        221184
transformer.layers.7.0.to_out.0.weight   [192, 384]         73728
transformer.layers.7.0.to_out.0.bias     [192]              192
transformer.layers.7.1.net.0.weight      [192]              192
transformer.layers.7.1.net.0.bias        [192]              192
transformer.layers.7.1.net.1.weight      [384, 192]         73728
transformer.layers.7.1.net.1.bias        [384]              384
transformer.layers.7.1.net.4.weight      [192, 384]         73728
transformer.layers.7.1.net.4.bias        [192]              192
transformer.layers.8.0.norm.weight       [192]              192
transformer.layers.8.0.norm.bias         [192]              192
transformer.layers.8.0.to_qkv.weight     [1152, 192]        221184
transformer.layers.8.0.to_out.0.weight   [192, 384]         73728
transformer.layers.8.0.to_out.0.bias     [192]              192
transformer.layers.8.1.net.0.weight      [192]              192
transformer.layers.8.1.net.0.bias        [192]              192
transformer.layers.8.1.net.1.weight      [384, 192]         73728
transformer.layers.8.1.net.1.bias        [384]              384
transformer.layers.8.1.net.4.weight      [192, 384]         73728
transformer.layers.8.1.net.4.bias        [192]              192
transformer.layers.9.0.norm.weight       [192]              192
transformer.layers.9.0.norm.bias         [192]              192
transformer.layers.9.0.to_qkv.weight     [1152, 192]        221184
transformer.layers.9.0.to_out.0.weight   [192, 384]         73728
transformer.layers.9.0.to_out.0.bias     [192]              192
transformer.layers.9.1.net.0.weight      [192]              192
transformer.layers.9.1.net.0.bias        [192]              192
transformer.layers.9.1.net.1.weight      [384, 192]         73728
transformer.layers.9.1.net.1.bias        [384]              384
transformer.layers.9.1.net.4.weight      [192, 384]         73728
transformer.layers.9.1.net.4.bias        [192]              192
transformer.layers.10.0.norm.weight      [192]              192
transformer.layers.10.0.norm.bias        [192]              192
transformer.layers.10.0.to_qkv.weight    [1152, 192]        221184
transformer.layers.10.0.to_out.0.weight  [192, 384]         73728
transformer.layers.10.0.to_out.0.bias    [192]              192
transformer.layers.10.1.net.0.weight     [192]              192
transformer.layers.10.1.net.0.bias       [192]              192
transformer.layers.10.1.net.1.weight     [384, 192]         73728
transformer.layers.10.1.net.1.bias       [384]              384
transformer.layers.10.1.net.4.weight     [192, 384]         73728
transformer.layers.10.1.net.4.bias       [192]              192
transformer.layers.11.0.norm.weight      [192]              192
transformer.layers.11.0.norm.bias        [192]              192
transformer.layers.11.0.to_qkv.weight    [1152, 192]        221184
transformer.layers.11.0.to_out.0.weight  [192, 384]         73728
transformer.layers.11.0.to_out.0.bias    [192]              192
transformer.layers.11.1.net.0.weight     [192]              192
transformer.layers.11.1.net.0.bias       [192]              192
transformer.layers.11.1.net.1.weight     [384, 192]         73728
transformer.layers.11.1.net.1.bias       [384]              384
transformer.layers.11.1.net.4.weight     [192, 384]         73728
transformer.layers.11.1.net.4.bias       [192]              192
transformer.layers.12.0.norm.weight      [192]              192
transformer.layers.12.0.norm.bias        [192]              192
transformer.layers.12.0.to_qkv.weight    [1152, 192]        221184
transformer.layers.12.0.to_out.0.weight  [192, 384]         73728
transformer.layers.12.0.to_out.0.bias    [192]              192
transformer.layers.12.1.net.0.weight     [192]              192
transformer.layers.12.1.net.0.bias       [192]              192
transformer.layers.12.1.net.1.weight     [384, 192]         73728
transformer.layers.12.1.net.1.bias       [384]              384
transformer.layers.12.1.net.4.weight     [192, 384]         73728
transformer.layers.12.1.net.4.bias       [192]              192
transformer.layers.13.0.norm.weight      [192]              192
transformer.layers.13.0.norm.bias        [192]              192
transformer.layers.13.0.to_qkv.weight    [1152, 192]        221184
transformer.layers.13.0.to_out.0.weight  [192, 384]         73728
transformer.layers.13.0.to_out.0.bias    [192]              192
transformer.layers.13.1.net.0.weight     [192]              192
transformer.layers.13.1.net.0.bias       [192]              192
transformer.layers.13.1.net.1.weight     [384, 192]         73728
transformer.layers.13.1.net.1.bias       [384]              384
transformer.layers.13.1.net.4.weight     [192, 384]         73728
transformer.layers.13.1.net.4.bias       [192]              192
transformer.layers.14.0.norm.weight      [192]              192
transformer.layers.14.0.norm.bias        [192]              192
transformer.layers.14.0.to_qkv.weight    [1152, 192]        221184
transformer.layers.14.0.to_out.0.weight  [192, 384]         73728
transformer.layers.14.0.to_out.0.bias    [192]              192
transformer.layers.14.1.net.0.weight     [192]              192
transformer.layers.14.1.net.0.bias       [192]              192
transformer.layers.14.1.net.1.weight     [384, 192]         73728
transformer.layers.14.1.net.1.bias       [384]              384
transformer.layers.14.1.net.4.weight     [192, 384]         73728
transformer.layers.14.1.net.4.bias       [192]              192
transformer.layers.15.0.norm.weight      [192]              192
transformer.layers.15.0.norm.bias        [192]              192
transformer.layers.15.0.to_qkv.weight    [1152, 192]        221184
transformer.layers.15.0.to_out.0.weight  [192, 384]         73728
transformer.layers.15.0.to_out.0.bias    [192]              192
transformer.layers.15.1.net.0.weight     [192]              192
transformer.layers.15.1.net.0.bias       [192]              192
transformer.layers.15.1.net.1.weight     [384, 192]         73728
transformer.layers.15.1.net.1.bias       [384]              384
transformer.layers.15.1.net.4.weight     [192, 384]         73728
transformer.layers.15.1.net.4.bias       [192]              192
transformer.layers.16.0.norm.weight      [192]              192
transformer.layers.16.0.norm.bias        [192]              192
transformer.layers.16.0.to_qkv.weight    [1152, 192]        221184
transformer.layers.16.0.to_out.0.weight  [192, 384]         73728
transformer.layers.16.0.to_out.0.bias    [192]              192
transformer.layers.16.1.net.0.weight     [192]              192
transformer.layers.16.1.net.0.bias       [192]              192
transformer.layers.16.1.net.1.weight     [384, 192]         73728
transformer.layers.16.1.net.1.bias       [384]              384
transformer.layers.16.1.net.4.weight     [192, 384]         73728
transformer.layers.16.1.net.4.bias       [192]              192
transformer.layers.17.0.norm.weight      [192]              192
transformer.layers.17.0.norm.bias        [192]              192
transformer.layers.17.0.to_qkv.weight    [1152, 192]        221184
transformer.layers.17.0.to_out.0.weight  [192, 384]         73728
transformer.layers.17.0.to_out.0.bias    [192]              192
transformer.layers.17.1.net.0.weight     [192]              192
transformer.layers.17.1.net.0.bias       [192]              192
transformer.layers.17.1.net.1.weight     [384, 192]         73728
transformer.layers.17.1.net.1.bias       [384]              384
transformer.layers.17.1.net.4.weight     [192, 384]         73728
transformer.layers.17.1.net.4.bias       [192]              192
transformer.layers.18.0.norm.weight      [192]              192
transformer.layers.18.0.norm.bias        [192]              192
transformer.layers.18.0.to_qkv.weight    [1152, 192]        221184
transformer.layers.18.0.to_out.0.weight  [192, 384]         73728
transformer.layers.18.0.to_out.0.bias    [192]              192
transformer.layers.18.1.net.0.weight     [192]              192
transformer.layers.18.1.net.0.bias       [192]              192
transformer.layers.18.1.net.1.weight     [384, 192]         73728
transformer.layers.18.1.net.1.bias       [384]              384
transformer.layers.18.1.net.4.weight     [192, 384]         73728
transformer.layers.18.1.net.4.bias       [192]              192
transformer.layers.19.0.norm.weight      [192]              192
transformer.layers.19.0.norm.bias        [192]              192
transformer.layers.19.0.to_qkv.weight    [1152, 192]        221184
transformer.layers.19.0.to_out.0.weight  [192, 384]         73728
transformer.layers.19.0.to_out.0.bias    [192]              192
transformer.layers.19.1.net.0.weight     [192]              192
transformer.layers.19.1.net.0.bias       [192]              192
transformer.layers.19.1.net.1.weight     [384, 192]         73728
transformer.layers.19.1.net.1.bias       [384]              384
transformer.layers.19.1.net.4.weight     [192, 384]         73728
transformer.layers.19.1.net.4.bias       [192]              192
transformer.layers.20.0.norm.weight      [192]              192
transformer.layers.20.0.norm.bias        [192]              192
transformer.layers.20.0.to_qkv.weight    [1152, 192]        221184
transformer.layers.20.0.to_out.0.weight  [192, 384]         73728
transformer.layers.20.0.to_out.0.bias    [192]              192
transformer.layers.20.1.net.0.weight     [192]              192
transformer.layers.20.1.net.0.bias       [192]              192
transformer.layers.20.1.net.1.weight     [384, 192]         73728
transformer.layers.20.1.net.1.bias       [384]              384
transformer.layers.20.1.net.4.weight     [192, 384]         73728
transformer.layers.20.1.net.4.bias       [192]              192
transformer.layers.21.0.norm.weight      [192]              192
transformer.layers.21.0.norm.bias        [192]              192
transformer.layers.21.0.to_qkv.weight    [1152, 192]        221184
transformer.layers.21.0.to_out.0.weight  [192, 384]         73728
transformer.layers.21.0.to_out.0.bias    [192]              192
transformer.layers.21.1.net.0.weight     [192]              192
transformer.layers.21.1.net.0.bias       [192]              192
transformer.layers.21.1.net.1.weight     [384, 192]         73728
transformer.layers.21.1.net.1.bias       [384]              384
transformer.layers.21.1.net.4.weight     [192, 384]         73728
transformer.layers.21.1.net.4.bias       [192]              192
transformer.layers.22.0.norm.weight      [192]              192
transformer.layers.22.0.norm.bias        [192]              192
transformer.layers.22.0.to_qkv.weight    [1152, 192]        221184
transformer.layers.22.0.to_out.0.weight  [192, 384]         73728
transformer.layers.22.0.to_out.0.bias    [192]              192
transformer.layers.22.1.net.0.weight     [192]              192
transformer.layers.22.1.net.0.bias       [192]              192
transformer.layers.22.1.net.1.weight     [384, 192]         73728
transformer.layers.22.1.net.1.bias       [384]              384
transformer.layers.22.1.net.4.weight     [192, 384]         73728
transformer.layers.22.1.net.4.bias       [192]              192
transformer.layers.23.0.norm.weight      [192]              192
transformer.layers.23.0.norm.bias        [192]              192
transformer.layers.23.0.to_qkv.weight    [1152, 192]        221184
transformer.layers.23.0.to_out.0.weight  [192, 384]         73728
transformer.layers.23.0.to_out.0.bias    [192]              192
transformer.layers.23.1.net.0.weight     [192]              192
transformer.layers.23.1.net.0.bias       [192]              192
transformer.layers.23.1.net.1.weight     [384, 192]         73728
transformer.layers.23.1.net.1.bias       [384]              384
transformer.layers.23.1.net.4.weight     [192, 384]         73728
transformer.layers.23.1.net.4.bias       [192]              192
transformer.layers.24.0.norm.weight      [192]              192
transformer.layers.24.0.norm.bias        [192]              192
transformer.layers.24.0.to_qkv.weight    [1152, 192]        221184
transformer.layers.24.0.to_out.0.weight  [192, 384]         73728
transformer.layers.24.0.to_out.0.bias    [192]              192
transformer.layers.24.1.net.0.weight     [192]              192
transformer.layers.24.1.net.0.bias       [192]              192
transformer.layers.24.1.net.1.weight     [384, 192]         73728
transformer.layers.24.1.net.1.bias       [384]              384
transformer.layers.24.1.net.4.weight     [192, 384]         73728
transformer.layers.24.1.net.4.bias       [192]              192
mlp_head.weight                          [100, 192]         19200
mlp_head.bias                            [100]              100
==================================================================
Total Parameters: 11139844
\end{lstlisting}
}


\end{appendices}

\end{document}
